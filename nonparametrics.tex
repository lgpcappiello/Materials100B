%Week 10, day 1: additional slides
\begin{frame}{Nonparametric Methods}
    \begin{itemize}
        \item The methods discussed thus far are all \textbf{parametric methods}.
        \begin{itemize}
            \item Parametric methods make a lot of assumptions about model parameters, such as distributional assumptions.
        \end{itemize}
        \item We will introduce some \textbf{nonparametric methods}.
        \begin{itemize}
            \item These require less restrictive assumptions.
        \end{itemize}
    \end{itemize}
\end{frame}

\begin{frame}{When Are Nonparametric Methods Useful?}
    \begin{itemize}
        \item The data are \textit{nominal} or \textit{ordinal} in nature (ordered or unordered categorical). 
        \item The parametric assumptions for a test are not satisfied.
        \begin{itemize}
            \item When we used t-tests, we needed to assume normality. 
            \item When the populations are not normally distributed, we should use a nonparametric test instead.
        \end{itemize}
    \end{itemize}
\end{frame}

\begin{frame}{When Are Nonparametric Methods Useful?}
    \begin{itemize}
        \item Even if a parametric method is appropriate, nonparametric methods tend to give similar results.
        \begin{itemize}
            \item We don't always use them because they can be more complex to interpret.
            \item Nonparametric methods may also have slightly lower power.
        \end{itemize}
    \end{itemize}
\end{frame}

\begin{frame}{Example: The Sign Test}
    \begin{itemize}
        \item A company is producing a new orange juice. 
        \item They want to know whether people prefer their orange juice or a competitor's orange juice.
        \item 12 individuals were given unmarked samples of orange juice in a random order.
        \item Each individual identified with orange juice they preferred. 
        \item We want to know whether preferences for the two juices are equal.
    \end{itemize}
\end{frame}

\begin{frame}{Example: Hypotheses}
    The hypotheses are
    \begin{align*}
        H_0\text{: }& p=0.5 \quad \text{No difference in preference.} \\
        H_A\text{: }& p\ne0.5 \quad \text{One product is preferred more than the other.} 
    \end{align*}
\end{frame}

\begin{frame}{Example: The Data}
    \centering
    \begin{tabular}{rl}
        \textbf{Individual} & \textbf{Brand Preference} \\
        1 & Tropical Orange \\ 
        2 & Tropical Orange \\ 
        3 & Citrus Valley \\ 
        4 & Tropical Orange \\ 
        5 & Tropical Orange \\ 
        6 & Tropical Orange \\ 
        7 & Tropical Orange \\ 
        8 & Tropical Orange \\ 
        9 & Citrus Valley \\ 
        10 & Tropical Orange \\ 
        11 & Tropical Orange \\ 
        12 & Tropical Orange \\ 
    \end{tabular}
\end{frame}

\begin{frame}{Sign Test P-Values}
    Under $H_0$, the number of $+$ signs is distributed Binomial(n, p).

    \vspace{12pt}The p-value for the sign test with hypotheses
    \begin{align*}
        H_0\text{: }& p=0.5 \quad \text{No difference in preference.} \\
        H_A\text{: }& p\ne0.5 \quad \text{One product is preferred more than the other.} 
    \end{align*}
    is 
    \[
        2 \times P(\text{number of + signs } \le \text{ observed number of + signs})
    \]
\end{frame}

\begin{frame}{Example: Orange Juice}
    Find the p-value for the orange juice example. What are the conclusions for this test?
\end{frame}

\begin{frame}{The Sign Test}
    For a Binomial($n=12$, $p=0.5$) distribution,
    \begin{table}[h]
        \centering
        \begin{tabular}{cc | cc}
            \hline
            Number of + & Probability & Number of + & Probability \\
            \hline
            0 & 0.0002 & 7 & 0.1934 \\
            1 & 0.0029 & 8 & 0.1208 \\
            2 & 0.0161 & 9 & 0.0537 \\
            3 & 0.0537 & 10 & 0.0161 \\
            4 & 0.1208 & 11 & 0.0029 \\
            5 & 0.1934 & 12 & 0.0002 \\ 
            6 & 0.2256 & & \\
            \hline 
        \end{tabular}
    \end{table}
\end{frame}

\begin{frame}{The Sign Test}
    \begin{itemize}
        \item The \textbf{sign test} is the nonparametric version of the \textit{one-sample t-test}.
        \item The previous example showed it used for proportions.
        \item We can also use this test for measures of center.
        \begin{itemize}
            \item For nonparametric tests, we often use the median as a measure for center. 
        \end{itemize}
    \end{itemize}
\end{frame}

\begin{frame}{Hypothesis Tests About a Median}
    \begin{align*}
        H_0\text{: Median}&=M_0 \\
        H_A\text{: Median}&\ne M_0
    \end{align*}
    \begin{itemize}
        \item Recall: the median splits the data in half so that 50\% of the values fall above and 50\% fall below.
        \item We apply the sign test by using a + sign when the value is above the hypothesized median and a - sign when it is below. 
        \item The computations are otherwise the same.
    \end{itemize}
\end{frame}

\begin{frame}{Large Sample Approach}
    \begin{itemize}
        \item Recall: we can use a normal distribution to approximate the binomial distribution.
        \item For large values of $n$, a Binomial(n, p) is well-approximated by
        \[
            N(\mu = np, \sigma=\sqrt{np(1-p)})
        \]
    \end{itemize}
\end{frame}

\begin{frame}{Example}
    We want to know about the median home price in St. Louis, MO.
    \begin{align*}
        H_0\text{: Median}&=\$75,000 \\
        H_A\text{: Median}&\ne \$75,000
    \end{align*}
    \begin{itemize}
        \item There is a sample of $n=62$ sales.
        \item 37 had prices above $\$75,000$.
        \item 23 had prices below $\$75,000$.
        \item 2 had prices exactly equal to $\$75,000$.
    \end{itemize}
\end{frame}

\begin{frame}{Wilcoxon Signed-Rank Test}
    The \textbf{Wilcoxon signed-rank test} is the nonparametric alternative to the paired t-test.
    \begin{itemize}
        \item Requires numeric data.
        \item The t-test requires that the differences are normally distributed.
        \item The Wilcoxon signed-rank test does not have these same requirements. 
    \end{itemize}
\end{frame}

\begin{frame}{Example}
    A manufacturing firm is examining task completion times for two production methods.
    \begin{table}[h]
        \centering
        \begin{tabular}{cccc}
            \hline
            Worker & Method 1 & Method 2 & Difference \\ 
            \hline
            1 & 10.2 & 9.5 & 0.7 \\
            2 & 9.6 & 9.8 & -0.2 \\
            3 & 9.2 & 8.8 & 0.4 \\
            4 & 10.6 & 10.1 & 0.5 \\
            5 & 9.9 & 10.3 & -0.4 \\
            6 & 10.2 & 9.3 & 0.9 \\
            7 & 10.6 & 10.5 & 0.1 \\
            8 & 10.0 & 10.0 & 0 \\
            9 & 11.2 & 10.6 & 0.6 \\
            10 & 10.7 & 10.2 & 0.5 \\
            11 & 10.6 & 9.8 & 0.8 \\ 
            \hline 
        \end{tabular}
    \end{table}
\end{frame}

\begin{frame}{Wilcoxon Signed-Rank Test}
    \begin{enumerate}
        \item Take the difference.
        \item Take the absolute value of the difference.
        \item Rank the absolute values of the differences.
        \item Reapply the signs from the differences to these ranks.
        \item Sum the signed ranks.
    \end{enumerate}
\end{frame}

\begin{frame}{Example}
    Find the sum of the signed ranks for the manufacturing firm data. 
\end{frame}

\begin{frame}{Wilcoxon Signed-Rank Test}
    Let $T$ denote the sum of the signed-rank values.
    
    \vspace{12pt}For data with at least 10 pairs, $T$ is well-approximated by
    \[
        N\left(\mu = 0, \sigma = \sqrt{\frac{n(n+1)(2n+1)}{6}}\right)
    \]
    where $n$ is the number of pairs with a nonzero difference. 
\end{frame}

\begin{frame}{Example}
    Find the normal distribution that approximates $T$ for the manufacturer data. 
\end{frame}

\begin{frame}{Wilcoxon Signed-Rank Test}
    Then the test statistic $z$ is
    \[
        z = \frac{T - \mu}{\sigma}
    \]
    
    \vspace{12pt}Find the test statistic for the manufacturer data. 
\end{frame}
