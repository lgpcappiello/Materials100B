\begin{frame}{The Randomized Block Design}
    \begin{itemize}
        \item The completely randomized design is best used when experimental units are \textit{homogeneous} (the same/similar).
        \item It also allows us to examine only one factor (the treatment or groups).
        \item Any other variability in the response gets lumped in with experimental error.
    \end{itemize}
\end{frame}

\begin{frame}{The Randomized Block Design}
    \begin{itemize}
        \item Sometimes units are not at all homogeneous.
        \item Typically we aren't interested in this source of variation.
        \item Instead, we want to control for it.
        \item E.g., if I'm looking at the impact of growth hormones on rats, I may want to control for differences between males and females.
    \end{itemize}
\end{frame}

\begin{frame}{The Randomized Block Design}
    We isolate this additional information using a \textbf{randomized block design}.
    \begin{itemize}
        \item We are still interested in comparing $k$ treatment means.
        \item Now we will also have $b$ blocks.
        \item Each block should be made up of homogeneous experimental units.
        \item We will have $n = b\times k$ observations.
    \end{itemize}
\end{frame}

\begin{frame}{The ANOVA for Randomized Block Designs}
    \begin{itemize}
        \item We now have two factors: treatments/groups and blocks.
        \item Each will affect the response.
        \item This is sometimes called a two-way ANOVA.
    \end{itemize}
\end{frame}

\begin{frame}{Sum of Squares for Randomized Block Designs}
    The total sum of squares is now partitioned into three sources of variation:
    \[
        \text{SSTotal} = \text{SSG} + \text{SSB} + \text{SSE}
    \]
    \begin{itemize}
        \item SSG: sum of squares, groups
        \item SSB: sum of squares, blocks
        \item SSE: sum of squares, error
    \end{itemize}
\end{frame}

\begin{frame}{Degrees of Freedom for Randomized Block Design}
    Each source of variation has an accompanying degrees of freedom:
    \begin{itemize}
        \item $df_{\text{groups}} = k-1$
        \item $df_{\text{blocks}} = b-1$
        \item $df_{\text{error}} = (k-1)(b-1)$
        \item $df_{\text{total}} = n-1 = b\times k-1$
    \end{itemize}
    The mean square for each source of variation is the sum of squares divided by its degrees of freedom.
\end{frame}

\begin{frame}{ANOVA: Randomized Block Design}
    \begin{table}[]
        \centering\setlength{\extrarowheight}{10pt}
        \begin{tabular}{lllll}
            \hline
            Source  & df & SS & MS & F \\
            \hline
            Group   & $k-1$ & $SSG$ & $MSG = \frac{SSG}{k-1}$ & MSG/MSE \\
            Blocks  & $b-1$ & $SSB$ & $MSG = \frac{SSG}{b-1}$ & MSG/MSE \\
            Error   & $(k-1)(b-1)$ & $SSE$ & $MSE =\frac{SSE}{(k-1)(b-1)}$\\
            \hline
            Total   & $n-1 = bk-1$ & $SST$ \\
            \hline
        \end{tabular}
    \end{table}
\end{frame}

\begin{frame}{Example}
    \begin{itemize}
        \item The cost of a cellphone minute varies drastically depending on the number of minutes per month used by the customer.
        \item A consumer watchdog group decided to compare the average costs for four cellular phone companies using three different usage levels as blocks.
        \item The monthly costs (in dollars) were computed for peak-time callers at low (20 minutes per month), middle (150 minutes per month), and high (1000 minutes per month).
        \item We want to construct the analysis of variance table for this experiment.
    \end{itemize}
\end{frame}

\begin{frame}{Example}
    The data is shown below.
    \begin{table}[h]
        \centering
        \begin{tabular}{l ccccc}
             & \multicolumn{4}{c}{Company} & \\
             \cline{2-5}
            Usage Level & A & B & C & D & Total \\ 
            \hline
            Low & 27 & 24 & 31 & 23 & 105 \\
            Middle & 68 & 76 & 65 & 67 & 276 \\
            High & 308 & 326 & 213 & 300 & 1246 \\
            \hline
            Total & 403 & 426 & 408 & 390 & 1627 \\
        \end{tabular}
    \end{table}
    Using this data, we can use a computer to find SSTotal$ = 189,798.9167$, SSGroup$= 222.25$, and SSBlock$= 189,335.1667$. 
\end{frame}

\begin{frame}{Hypothesis Tests}
    Now we have two $F$ values to worry about. These correspond to tests regarding treatment means
    \begin{align*}
       H_0&: \quad \text{No difference among } k \text{ group means.} \\
       H_A&: \quad \text{At least one pair of group means is not equal.}
    \end{align*}
    and block means
    \begin{align*}
        H_0&: \quad \text{No difference among } b \text{ block means.} \\
        H_A&: \quad \text{At least one pair of block means is not equal.}
    \end{align*}
\end{frame}

\begin{frame}{Two-Way ANOVA}
    The intuition behind partitioning variance and the approach to the F-tests are exactly the same as with the one-way ANOVA that we learned last week!
\end{frame}

\begin{frame}{Example}
    \begin{enumerate}
        \item Write the hypotheses for the ANOVA from the previous example.
        \item What can we conclude?
    \end{enumerate}
\end{frame}
