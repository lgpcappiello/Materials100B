\begin{frame}{Multiple Regression}
    \begin{itemize}
        \item An outcome may be simultaneously influenced by many variables.
        \item Think back to our randomized block and factorial designs.
        \item Multiple regression extends this idea into the regression framework.
        \item We will extend the simple linear regression to include many predictor variables.
    \end{itemize}
\end{frame}

\begin{frame}{Consider}
    We will work with a data set that contains patient data for a clinical trial with 1000 participants. 
    \begin{itemize}
        \item Each patient entered the study with high LDL cholesterol.
        \item Patients were randomly assigned to either a medication to manage their cholesterol or to a placebo. 
        \item We can always examine treatment and change in LDL cholesterol.
        \item But what about other variables? 
    \end{itemize}
\end{frame}

\begin{frame}{The Data}
    \begin{table}[h]
        \centering\small
        \begin{tabular}{lccccccccc}
            \hline
             & ldl.post & trt & sex & age & weight & sys.bp & dia.bp & income & ldl.pre \\
            \hline
            1 & 176   & 0 &  M & 47  &  186  &  119  &   64   & low & 178 \\
            2 & 191  &  0 &  M & 37  &  185  &  121  &   72   & med & 191 \\
            3 & 155  &  1 &  M & 48  &  208   & 106  &   71  & med & 203 \\
            4 & 123  &  1 &  F & 46  &  159   & 106  &   57  & med & 168 \\
            5 & 120  &  1 &  M & 37  &  117  &  100  &   74  & low & 168 \\
            6 & 134 &   1  & F & 38  &  228 &   128  &   71  & low & 190 \\
            \vdots & \vdots & \vdots & \vdots & \vdots & \vdots & \vdots &   \vdots & \vdots \\
            \hline
        \end{tabular}
    \end{table}
    How might weight impact the medication's effectiveness? What about blood pressure?
\end{frame}

\begin{frame}{Indicator and Categorical Variables as Predictors}
    We start by fitting a linear regression model using \texttt{trt} to predict \texttt{ldl.post - ldl.pre}.
    \begin{table}[h]
        \centering
        \begin{tabular}{lcccc}
            \hline
                        & Estimate & Std. Error & t value & Pr($>|t|$) \\
            \hline
            (Intercept) &  0.1588  & 0.1576  &  1.007  &  0.314   \\
            trt         & -48.1471 & 0.2196 & -219.216 &  $<$2e-16 \\
            \hline
        \end{tabular}
    \end{table}
    Write the regression line. Then, interpret the slope and intercept.
\end{frame}

\begin{frame}{Indicator and Categorical Variables as Predictors}
    We can also fit models using categorical variables with more than 2 levels. 
    \begin{itemize}
        \item The \texttt{income} variable has 3 levels: high, medium, and low.
        \item Suppose we want to know whether socioeconomic status is relevant to treatment outcomes.
    \end{itemize}
\end{frame}

\begin{frame}{Indicator and Categorical Variables as Predictors}
    \begin{table}[h]
        \centering
        \begin{tabular}{lcccc}
            \hline
                        & Estimate & Std. Error & t value & Pr($>|t|$) \\
            \hline
            (Intercept) & -23.6000   &  1.8949& -12.454  & $<$2e-16 \\
            incomelow   &  -0.9113   &  2.1937 & -0.415   & 0.678  \\ 
            incomemed   &  -1.7000   &  2.2986  &-0.740&    0.460 \\
            \hline
        \end{tabular}
    \end{table}
    \begin{itemize}
        \item Each row represents the relative difference for each level.
        \item Notice we are missing income:high. This is the \textbf{reference level} that the other variables are measured against.
        \item I let \texttt{R} choose the reference level, but we could pick any one of the income levels to act as the "default".
    \end{itemize}
\end{frame}

\begin{frame}{Example}
    None of the levels of income appear to be good predictors for our treatment outcomes, but let's think about how to use the regression equation.
\end{frame}

\begin{frame}{Predictors with Several Categories}
    \begin{itemize}
        \item When fitting a regression model with a categorical variable that has $k$ levels, standard software will provide a coefficient for $k − 1$ of them.
        \item For the level that does not receive a coefficient, this is the reference level. 
        \item The coefficients listed for the other levels are all considered relative to this reference level.
    \end{itemize}
\end{frame}

\begin{frame}{Including and Assessing Many Variables in a Model}
    \begin{itemize}
        \item The world is complex! More information is typically better information.
        \item If we have the ability to collect and use many variables, we should use them!
        \item This is the idea behind \textbf{multiple linear regression}.
    \end{itemize}
\end{frame}

\begin{frame}{Including and Assessing Many Variables in a Model}
    We want to construct a model for our cholesterol data using all of the variables simultaneously:
    \begin{align*}
        \hat{\texttt{ldl.post}} = \beta_0 &+ \beta_1 \times \texttt{trt} + \beta_2 \times \texttt{sex} + \beta_3 \times \texttt{age} \\
        &+ \beta_4 \times \texttt{weight} + \beta_5 \times \texttt{sys.bp} + \beta_6 \times \texttt{dia.bp} \\
        &+ \beta_7 \times \texttt{income}_{med} + \beta_8 \times \texttt{income}_{low} + \beta_9 \times \texttt{ldl.pre} 
    \end{align*}
\end{frame}

\begin{frame}{Estimating Parameters}
    We estimate $\beta_0, \beta_1, \dots, \beta_9$ the same way we did for our linear regression with only two parameters, by minimzing the sum of squared residuals
    \[
        SSE = \sum_{i=1}^n (y_i - \hat{y}_i)^2
    \]
    
    \vspace{18pt}But... this time we'll definitely use a computer.
\end{frame}

\begin{frame}{Example}
    \begin{table}[h]
        \centering
        \begin{tabular}{lcccc}
            \hline
                        & Estimate & Std. Error & t value & Pr($>|t|$) \\
            \hline
            (Intercept) & 26.50  & 1.773 &  14.95 & $<$ 2e-16 \\
            trt     &    -48.20  & 0.159 &-303.71 & $<$ 2e-16 \\
            sexM    &     -0.63  & 0.159 &  -3.99 & 7.23e-05 \\
            age     &      0.02  & 0.012  &  1.29 &  0.197     \\
            weight  &     -0.01  & 0.003 &  -2.45 &  0.015   \\
            sys.bp  &     -0.15  & 0.006 & -27.44 & $<$ 2e-16 \\
            dia.bp  &     -0.11  & 0.010 & -10.84 & $<$ 2e-16 \\
            incomelow  &  -0.23  & 0.227 &  -1.01 &  0.314     \\
            incomemed  &  -0.05 &  0.238 &  -0.19  & 0.848     \\
            ldl.pre    &   0.99  & 0.008 & 126.16 & $<$ 2e-16 \\
            \hline
        \end{tabular}
    \end{table}
    Write out the regression model. 
\end{frame}

\begin{frame}{Example}
    \begin{itemize}
        \item Interpret the coefficients corresponding to \texttt{sexM} and \texttt{age}. 
        \item Calculate the residual for the first patient. 
    \end{itemize}
    \begin{table}[h]
        \centering\small
        \begin{tabular}{lccccccccc}
            \hline
             & ldl.post & trt & sex & age & weight & sys.bp & dia.bp & income & ldl.pre \\
            1 & 176   & 0 &  M & 47  &  186  &  119  &   64   & low & 178 \\
            \hline
        \end{tabular}
    \end{table}
\end{frame}

\begin{frame}{Example}
    The estimated linear regression line for $\texttt{LDL.post} = \beta_0 + \beta_1 \texttt{trt}$ is 
    \[
        \hat{\texttt{LDL.post}}=175.0309 -49.0756\times\texttt{trt}.
    \]
    with $SE(b_1) = 0.6657$. 
    
    \vspace{12pt}Why is this different from the estimate and standard error for \texttt{trt} in the multiple regression model?
\end{frame}

\begin{frame}{Correlation Between Predictor Variables}
    \begin{itemize}
        \item We say the two predictor variables are \textbf{collinear} when they are correlated.
        \item This complicates model estimation.
        \item We can't always prevent collinearity, but we do want to control it.
        \item Ex: height and arm span give us essentially the same information. We wouldn't want to use both in a model.
    \end{itemize}
\end{frame}

\begin{frame}{Goodness-of-Fit}
    Recall that we used $R^2$ to determine the amount of variability explained by the model:
    \[
        R^2 = 1 - \frac{\text{variability in residuals}}{\text{total variability}} = 1- \frac{SS_{residuals}}{SS_{total}}
    \]
    We can continue to use this in multiple regression.
\end{frame}

\begin{frame}{Goodness-of-Fit}
    \begin{itemize}
        \item $R^2$ will always increase when we include more variables in the model.
        \item This is true even if the variables aren't very useful!
        \item We want a measure that will help us balance model efficacy with model size.
    \end{itemize}
\end{frame}

\begin{frame}{Adjusted $R^2$}
    \textbf{Adjusted} $\boldsymbol{R^2}$ is computed as
    \[
        R^2_{adj} = 1 - \frac{SS_{resid}/(n-k-1)}{SS_{total}/(n-1)}
    \]
    where $n$ is the number of observations and $k$ is the number of predictor variables in the model.
    
    \vspace{12pt} Note that $k$ includes the $p-1$ predictor variables for categorical variables with $p$ levels.
\end{frame}

\begin{frame}{Adjusted $R^2$}
    Notice that
    \begin{align*}
        R^2_{adj} = 1 - \frac{SS_{resid}}{SS_{total}}\times\frac{(n-1)}{(n-k-1)}
    \end{align*}
    and
    \[
        R^2 = 1 - \frac{SS_{resid}}{SS_{total}}
    \]
    Since $k \ge 1$, $R^2_{adj} < R^2$.
\end{frame}

\begin{frame}{Adjusted $R^2$}
    \begin{itemize}
        \item The idea here lies with degrees of freedom.
        \item We adjust $R^2$ based on model and error df.
        \item This balances efficacy and model size (what we wanted).
        \item This will also help us compare models.
    \end{itemize}
\end{frame}
